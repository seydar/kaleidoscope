\documentclass{article}
\newcommand{\degree}{\ensuremath{^\circ}}
\title{Immix Garbage Collection and Simulator Implemented in Ruby}
\author{Ari Browm and Bo Chen}
\date{\today}
\begin{document}
  \maketitle

  \section{Introduction}
  Immix garbage collector is one that takes advantage of both mark-sweep and
  semi-space garbage collection mechanisms, first introduced and published in
  \emph{Immix Garbage Collection: Fast Collection, Space Efficiency, and Mutator
  Locality}, by S. Blackburn and K. McKinley in 2007. It claims to achieve
  collector efficiency, heap space efficiency and mutator locality at once and
  outperforms traditional mark-sweep, mark-compact, and semi-space garbage
  collectors by 8 to 10 percent across benchmarks on 3 different architectures
  and all heap sizes. The research project detailed in this paper took the
  algorithm of the immix garbage collector and built it with Ruby to demonstrate
  its garbage collection mechanism. Our project also produced a GC simulator
  with the intent of being used for learning purposes. It features a bare-bones
  language, clean code, structural independence for the GC, testing facilities,
  and appropriate visualizations of memory.

  \section{Technical Solution}
  \subsection{The Immix Algorithm}
  The key components of the Immix garbage collector consists of first, the
  organization of the memory in contiguous blocks of lines; second, line and
  block granularity reclamation, and third, an opportunistic compaction
  mechanism. The garbage collection is triggered after the allocation has
  exhausted the heap. The immix collector performs a transitive closure by
  tracing through the object path and records the lives in which live objects
  are stored. Immix then performs a block-granularity sweep by scanning the line
  map to identify completely free blocks, which will be returned to the global
  pool and the partially free blocks which will be recycled for steady state
  allocation. In this phase, the thread-local allocator will bump allocate into
  holes (one or more contiguous unused lines) until all recycled blocks are
  exhausted, and allocation will then resume to allocate into the empty blocks
  till the exhaustion of the heap. \\

  Opportunistic Compaction is performed to eliminate fragmentation based on the
  number of holes, the liveness of the objects and the unused memory available
  within the blocks under inspection. Compaction is triggered when there are one
  or more recyclable blocks didn't used by the allocator or the previous
  collection failed yield enough free space. If the object is live, not pinned
  and there is available space, the live object in the compaction candidate
  block will then be opportunistically copied and allocated in the same way as
  the steady state allocation. If a object is pinned, it is marked as live and
  left unmoved.

  \subsection{Detail and Evaluation}
  The immix garbage collector treats memory as blocks of contiguous memory
  consisting of lines. The size of a line is 128B and a block size is 32KB. The
  lines correspond the the fine granularity and blocks correspond to the coarse
  granularity, where a block is also the finest granularity shared by threads.
  The immix garbage collector design favors unsynchronized thread-local
  activities over the synchronized global activities in order to achieve
  maximized parallelism and also good locality. Therefore, the size of block is
  closely related to the issue of space sharing, synchronization overhead and
  fragmentation. Smaller sizes contribute to better space sharing and lower
  fragmentation ratio at the cost of synchronization overhead, while larger
  block sizes leverage less synchronization overhead at the cost of poorer space
  sharing and higher fragmentation. \\

  Conservative line marking is employed as opposed to exact marking scheme.
  Before mark time, objects occupying space larger than a line is marked with a
  bit in their header during allocation as opposed to collection. At mark time,
  line mark for the last line of large objects and the extra line of small
  object spilling into another line. Before allocation, line marks are corrected
  by marking the first line of every hole. Small objects are marked by setting
  the line of their start address, and larger objects are marked all the lines
  within their size except their last line. Based on the observation that the
  majority of the objects have size smaller than a line, such technique is lot
  cheaper than exact line marking which has to obtain the object's type and
  iterate all the address range during marking, at the cost of worst case
  scenario in which one line is wasted for every hole.

  \section{Implementation}
  The project’s scope is to demonstrate the immix gc in action, with the memory
  organized in blocks with size of 200 bytes, mark-sweep and opportunistic
  compaction mechanism to fully simulate a single thread immix garbage collector
  implemented in ruby.

  \subsection{Problems Relating to Real-World Implementation}
  Our first implementation featured an actual immix garbage collector featured
  an allocator dealt with low-level details using <sys/mman> library. However,
  we have encountered inconsistent memory faults resulting from memory
  allocation on OSX and linux environment, resulting in garbage collector
  implemented in Ruby to fail to allocate the memory from the heap. The issue
  was alleviated after using malloc() to obtain memory for the purpose of
  bookkeeping of the status of memory at block granularity, we then encountered
  issues with the incorrect block address alignment in the block initialization
  process, with a failed assertion that the newly allocated block's starting
  address failed to be aligned at the multiple of the blocksize (boundary of
  block sizes) after alignment step in the initialization process. Such resulted
  in a faulty casting from an Address object (Address addr;) to an instance of
  Object (Object obj;) when used in the allocator portion of the implementation
  and resulted in memory faults, which hamper the success of the entire
  implementation from correctly tracking and managing objects for garbage
  collection.

  \subsection{Simulator}
  To alleviate the problems we had before, we decided to create a simulation
  environment where we could deterministically build and run a garbage collector
  with the ultimate goal of learning from it. This took the project from a
  real-world example to an education tool. The advantage of using our tool to
  build and study algorithms is that you can programmatically build test cases
  using the frontend of the Kaleidoscope language, examine the memory and GC
  functions, and continue running the program in bullet-time, if so desired. \\

  The language consists of an LALR(1) parser that then interprets the resultant
  AST and executes the accompanying Ruby code on-the-fly. In the early days of
  the project, the interpreter and simulation environment ended there, but
  changes were quickly made to include a fake memory object which is allocated
  from and treated as though it were real. The language implements floats and
  linked lists, along with list element access in O(n) time. We felt that these
  two datatypes were sufficient to showcase a garbage collectors abilities. \\

  We implemented a single-threaded version of the Immix GC as an example of how
  to use this system. The system was built to allow you to focus on only the
  garbage collector, abstracting away the details of the language and underlying
  memory. Thus, this makes it extremely easy to gather and provide statistics on
  both GC functions and memory access.

  \section{Results}
  The results of this project are a simulator and working garbage collector. As
  running the tests in the ‘test’ directory will show, the GC properly
  allocates, traces, and collects the memory contained in the blocks. It also
  successfully negotiates collections that occur in the middle of allocating
  linked lists. For instance, if a linked list allocation were to overflow a
  block full of unreachable objects, the GC would collect the block while
  maintaining the portion of the list that has already been allocated, and it
  would resume allocation at the next available spot, which in our example
  (test/list.rb) is the beginning of that same block. Our simulator also deals
  with allocating very large linked lists that span over a block. In the list
  construction code, it creates a root in the transitive object closure graph to
  the latest object in the list so that if collection does occur (and in our
  example it does indeed), the GC will be able to trace it and mark its parts.
  Due to the scale of our simulator, we chose to make line granularity the size
  of 1 byte, which, while defeating the improvement of speed, gives us a better
  understanding of how the GC works and is working. We focused instead on the
  aspects of the immix GC that makes is powerful: blocks, various granularity,
  and compaction at the block level. \\

  Our simulator itself has been quite successful for debugging the GC as we
  built it, which shows that the simulator has been a success and has met its
  purpose. The design of the test cases, which allow you to generate the code you
  want to run and then use the frontend of the language implementation, does
  full integration testing and offers extra support for memory and GC
  introspection. It is our hope that we continue to work on the simulator after
  this class and provide extra garbage collection algorithms to make this a more
  useful learning tool.

  \section{Improvement}
  Due to the scope of the project, we made a conscious decision to reduce the
  level of line granualarity. In future version, we would like to increase the
  granularity and visualization of lines so that the true power of the immix GC
  may be witnessed. \\

  As with all coding projects, a future improvement would be increased
  documentation and code encapsulation.

  \section{Conclusions}
  The immix garbage collection algorithm is a compromise between the slowness,
  fragmentation, and space efficiency of the Mark and Sweep algorithm and the
  speed, compactness, and space inefficiency of the Semi-Space Collection
  algorithm. It represents the future of GC algorithms and is a much needed
  update to a field subset that has been plateaued for many years. \\

  As a result of the past 40 years of garbage collection research, students
  fail to fully comprehend how they work and do not have a suitable environment
  in which to learn besides writing a collector themselves. This is, naturally,
  an inefficient and often unsuccessful manner of learning. Our simulator fills
  this gap in the educator's toolbox and provides a framework for studying
  garbage collectors in a tame, simplified language environment. It is better
  than purely emulated memory because it offers the facilities of a
  general-purpose programming language.

  \begin{thebibliography}{1}
    \bibitem{immix} S.M. Blackburn and K. S. Mckinley,
    \emph{Immix: A Mark-Region Garbage Collector
    with Space Efficiency, Fast Collection, and Mutator Performance},
    Australian National Univ., ACM, 2007.

    \bibitem{multi} S. Marlow and S.P. Jones,
    \emph{Multicore Garbage Collection with Local Heaps},
    Microsoft Research, Cambridge, UK, 2010.

    \bibitem{jones} R. Jones,
    \emph{Garbage Collection: Algorithms for Automatic Dynamic Memory
    Management},
    1996.

    \bibitem{other} R. Jones,
    \emph{Garbage Collection Handbook: Art of Automatic Memory Management},
    2012.

    \bibitem{code} http://github.com/seydar/kaleidoscope
  
  \end{thebibliography}

\end{document}

